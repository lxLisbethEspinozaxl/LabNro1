\section{Actividad No 09 – SubConsultas} 
		
\begin{enumerate}[1.]
	\item El departamento de Recursos Humanos requiere una consulta que pregunte al usuario por el Apellido del empleado, Luego la consulta deber\'a mostrar los Apellidos y Fecha de Contrataci\'on de todos los empleados del mismo departamento excluyendo o con excepción del empleado el cual ha sido proporcionado su apellido reporte que muestre las direcciones de todos los departamentos.
	\\
	\\-- para el desarrollo se cambio apellido por id de empleado
	\\
	\\DECLARE @depid INT;
	\\DECLARE @empid INT;
	\\
	\\--  se procede a leer el id de empleado
	\\
	\\SET @empid = 110
	\\
	\\--obteniendo id de departamento de empleado
	\\SET @depid = (SELECT emp.department\_id
	\\			FROM employees as emp
	\\			WHERE emp.employee\_id=@empid);
	\\
	\\--todos los empleados del mismo departamento excluyendo al empleado ingresado anteriormente
	\\SELE	CT emp.employee\_id,
	\\		emp.last\_name,
	\\		emp.hire\_date,
	\\		emp.department\_id
	\\FROM employees AS emp 
	\\WHERE emp.department\_id = @depid 
	\\AND emp.employee\_id !=@empid;\\

	\begin{center}
	\includegraphics[width=17cm]{./Imagenes/Actividad9-Ejercicio01} 
	\end{center}

	\item Crear un reporte que muestre el No del Empleado, Apellidos y Salarios de todos los empleados que tienen un salario superior al promedio de salarios de todos los empleados. Ordenar los resultados por el Salario de forma ascendente.
	\\
	\\--Se considera 'N de empleado' como 'id de empleado'
	\\--Obteniendo promedio de salario
	\\
	\\DECLARE @prom DECIMAL (8,2); -- Variable de promedio
	\\SET @prom = (SELECT AVG(salary) FROM employees);
	\\
	\\--Todos los empleados con sueldo superior al promedio
	\\SELECT emp.employee\_id,
	\\		emp.last\_name,
	\\		emp.salary
	\\		FROM employees AS emp
	\\WHERE emp.salary > @prom;\\

	\begin{center}
	\includegraphics[width=17cm]{./Imagenes/Actividad9-Ejercicio02} 
	\end{center}

	\item Realizar un reporte que muestre el No de Empleado y Apellidos de todos los empleados quienes trabajan en el departamento de cualquier empleado que su apellido contenga la letra ‘u’.
	\\
	\\--Se considera 'N de empleado' como 'id de empleado'
	\\--Obtener los id de departamentos de los empleados que contengan 'u' en su apellido
	\\
	\\SELECT DISTINCT department\_id
	\\FROM employees
	\\WHERE last\_name LIKE '\%u\%'
	\\
	\\--Obtener todos los empleados que laboren en alguno de los departamentos hallados anteriormente
	\\SELECT emp.employee\_id,
	\\		emp.last\_name,
	\\		emp.department\_id
	\\		FROM employees AS emp
	\\JOIN (SELECT DISTINCT department\_id
	\\		FROM employees
	\\		WHERE last\_name LIKE '\%u\%' ) AS depid
	\\		ON emp.department\_id=depid.department\_id\\

	\begin{center}
	\includegraphics[width=17cm]{./Imagenes/Actividad9-Ejercicio03} 
	\end{center}

	\item El departamento de Recursos Humanos requiere un reporte que muestre los Apellidos, No de Departamento y Puestos de los empleados cuya locación de departamento es 1700.
	\\
	\\SELECT emp.last\_name,
	\\		emp.department\_id,
	\\		dep.location\_id
	\\		FROM employees as emp
	\\JOIN departments as dep
	\\	ON emp.department\_id=dep.department\_id
	\\	WHERE dep.location\_id=1700;\\

	\begin{center}
	\includegraphics[width=17cm]{./Imagenes/Actividad9-Ejercicio04} 
	\end{center}

	\item Modificar la consulta anterior de forma que el usuario pueda introducir el No de locaci\'on.
	\\
	\\DECLARE @locid INT;
	\\SET @locid = 1700;
	\\SELECT emp.last\_name,
	\\		emp.department\_id,
	\\		dep.location\_id
	\\		FROM employees as emp
	\\JOIN departments as dep
	\\		ON emp.department\_id=dep.department\_id
	\\		WHERE dep.location\_id=@locid; \\

	\begin{center}
	\includegraphics[width=17cm]{./Imagenes/Actividad9-Ejercicio05} 
	\end{center}

	\item Crear un reporte para el departamento de Recursos Humanos que muestre los Apellidos y Salarios de todos los empleados cuyo Administrador apellide ‘King’.
	\\
	\\--conseguir id de empleado que lleven como apellido KING
	\\SELECT employee\_id,
	\\		last\_name
	\\		FROM employees
	\\		WHERE last\_name='KING';
	\\
	\\--conseguir id de departamentos que coincidan con el manager\_id con employee\_id
	\\
	\\SELECT dep.department\_id
	\\			FROM departments AS dep
	\\JOIN (SELECT employee\_id,
	\\			last\_name
	\\			FROM employees
	\\			WHERE last\_name='KING') as manking
	\\ON dep.manager\_id=manking.employee\_id
	\\
	\\--Obtener los apellidos y salarios de empleados que tengan como id de departamneto el/los id de departamento hallados \\anteriormente
	\\SELECT emp.last\_name,
	\\		emp.salary
	\\		FROM employees AS emp
	\\JOIN (SELECT dep.department\_id
	\\		FROM departments AS dep
	\\JOIN (SELECT employee\_id,
	\\		last\_name
	\\		FROM employees	
	\\		WHERE last\_name='KING') AS manking
	\\	ON dep.manager\_id = manking.employee\_id) AS depking
	\\ON emp.department\_id=depking.department\_id; \\

	\begin{center}
	\includegraphics[width=17cm]{./Imagenes/Actividad9-Ejercicio06} 
	\end{center}

	\item Crear un reporte para el departamento de Recursos Humanos que muestre el No de Departamento, Apellidos, Puestos de todos los empleados en el departamento ‘Executive’.
	\\
	\\SELECT * from employees where department\_id=90;
	\\SELECT * from jobs;
	\\SELECT * from departments where department\_name='executive';
	\\
	\\--consiguiendo empleados con nombre de puesto
	\\SELECT emp.department\_id,
	\\	emp.last\_name,
	\\	jobs.job\_title
	\\	FROM employees AS emp
	\\JOIN jobs
	\\ON emp.job\_id=jobs.job\_id;
	\\
	\\--Conseguir a los empleado con departamento categoria Executive
	\\SELECT empnomjob.department\_id,
	\\	empnomjob.last\_name,
	\\	empnomjob.job\_title
	\\	FROM departments
	\\JOIN (SELECT emp.department\_id,
	\\	emp.last\_name,
	\\	jobs.job\_title	
	\\	FROM employees AS emp
	\\JOIN jobs
	\\	ON emp.job\_id=jobs.job\_id) AS empnomjob
	\\	ON empnomjob.department\_id=departments.department\_id
	\\WHERE department\_name='executive';\\

	\begin{center}
	\includegraphics[width=17cm]{./Imagenes/Actividad9-Ejercicio07} 
	\end{center}

\end{enumerate}
